\begin{acknowledgements} \index{謝誌}
\index{ncuthesis 環境!acknowledgements}

在我的求學生涯中,有很多的老師。不過,讓我得益最多的還是這兩年多的碩士生涯。

還記得剛剛來台灣的時候,半個人都不認識,憑著老師一封電郵就來台灣唸書,實在是人生路不熟。 這兩年裡面,最感謝的人就是我的指導教授,陳俞融老師。 雖然老師對我並沒有好臉色,每次我報告完總是一副“奇怪,你怎麼又離題了”的樣子;但是我知道責之深愛之切,每次打開老師的面書,就會知道:喔,這次老師又要生我的氣了。

在整個碩士生涯中,我學會最多的就是如何篩選一篇相關的文獻,怎樣和我自己的論文相比較,從而得出究竟這篇文獻是否適用的過程。從一開始不會把整篇文獻看完,到後來認真地仔細地看別人引用的文章,再到後來自己寫出來的時候該如何引用。或者我的論文內容並不充實,但我認為碩士課程需要學習的就是如何有辨別文獻的相關性。並不是以偏概全,嘩眾取寵,而是只把自己有把握的地方撰寫出來。

本論文得以完成,實在是一件非常不容易的事情。由於本人英文寫作不佳,所以感謝好友Jess幫忙修改英文句子。謝謝學長samuel教我使用Latex,謝謝實驗室的學長豆花教導我做實驗,學弟妹謝妮恩,蘇映全,在同步輻射期間的幫忙。I also wants to thank Angela, Guillermo and Michel for your valuable comments on my thesis, which actually helps. Finally, thanks molview.org for providing an open source tool to creat 3D chemical models conviniently to make 3D diagrams.

\end{acknowledgements} 
