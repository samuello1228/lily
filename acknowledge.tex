\begin{acknowledgements} \index{謝誌}
\index{ncuthesis 環境!acknowledgements}

在我的求學生涯中,有很多的老師。不過,讓我得益最多的還是這兩年多的碩士生涯。

還記得剛剛來台灣的時候,半個人都不認識,憑著老師一封電郵就來台灣唸書,實在是人生路不熟。 這兩年裡面,最感謝的人就是我的指導教授,陳俞融老師。 雖然老師對我並沒有好臉色,每次我報告完總是一副“奇怪,你怎麼又離題了”的樣子;但是我知道責之深愛之切,每次打開老師的面書,就會知道:喔,這次老師又要生我的氣了。

本論文得以完成,實在是一件非常不容易的事情。由於本人英文寫作不佳,所以感謝Jess幫忙修改英文句子。謝謝學長samuel教我使用Latex,謝謝實驗室的學長豆花教導我做實驗,學弟妹謝妮恩,蘇映全,在同步輻射期間的幫忙。最後,還要感謝angela和geilermo教授的寶貴意見,令我獲益良多。

\end{acknowledgements} 
