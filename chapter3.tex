\chapter{\protect Experimental Results of CH$_4$ + NH$_3$ ice mixtures}
According to Grundy et al. (2016), CH$_4$ from Pluto may accumulate by cold-trapping, onto surface of Charon. The amount of CH$_4$ varies throughout the surface of Charon because it depends on duration of temperature below 25 K. The duration depends on diurnal motion and thermal inertia of Charon. With a tilted axis of 112 degrees to the ecliptic, higher concentration of CH$_4$ will accumulate at the pole (see chapter 1 for details). Therefore, we investigate different concentrations of CH$_4$+NH$_3$ ice mixtures and answer several questions: Will different concentration of CH$_4$ mix with high concentration of ammonia observed on crater position and throughout the surface of Charon (Grundy et al. 2016) have structure difference in accumulation of tholin? Are there variations of photo-products when concentration of CH$_4$ differ during warm-up? Since both EUV and VUV irradiation irradiates onto Charon, are there any differences when we change the photon source from VUV to EUV to irradiate the ice mixtures?\\

The main source to irradiate the dark side of Charon is Lyα reflected by interplanetary medium (Grundy 2016). Other sources such as the energetic ions in solar wind, consists of mainly H$^+$, He$^+$, He$^{++}$ and O$^{2+}$ etc are originated from solar corona or IPM. These ions would also reflect solar irradiation to the dark side of Charon. Among these irradiations, we picked He II irradiation because He II is 3 – 20 times more intense then He I during a solar flare. As it varies, it is difficult to estimate the dose onto Charon. Besides, electronic flux is also present in solar wind but it is one order of magnitude lower than proton flux. The flux for energetic electrons observed at the 1 A. U. position is available (http://www.swpc.noaa.gov/products/goes-electron-flux). Although electron flux is much less important than Lyα, and their flux varies, we also compare the electron irradiation experiment done by Kim and Kaiser (2011) on CH$_4$+NH$_3$ ice mixtures in this chapter. \\

When Charon is shine by direct sun light, the surface temperature increases and deliver the heat to the poles by conduction. From the model of Grundy et al. (2016), the surface temperature of the pole area would increase to 60 K that the heating rate depends on the thermal conductivity of Charon. To demonstrate the heating process, we warmup our ice mixture with a heating rate 1 K/min and monitor the ice by both QMS and scanning IR spectra with 5 K intervals. We will look into whether there are new species formed during warmup and monitor the gas phase desorption.\\

Finally, in this chapter, after we focus on the concentration effect of CH$_4$ on photo-products, photon energy effects, species detected during warmup phases, we present the residues accumulated by irradiating CH$_4$+NH$_3$ ice mixtures with different ratios. Since both tholin formed on Titan and Charon has similar colour, we also compare the IR spectra of MDHL, NSRRC with different configurations with the residues on Titan with experiments done by Imanaka et al.\\
\section{3.1.1 The concentration effect of CH$_4$ on production of C$_2$H$_6$ and CN$^-$}
We first look into the concentration effects of CH$_4$ by irradiation by VUV irradiation. Before and after deposition, we scanned an IR spectrum and plotted the absorbance of the ice mixtures. figure 3.1.1 plots the absorbance of the CH$_4$+NH$_3$ ice mixtures in different ratios. Due to the ice thickness, the infra-red spectrum of CH$_4$+NH$_3$ = 3:2 consist of 900 ML of CH$_4$ and 600 ML of NH$_3$ is tilted due to interference. Since the amount of ammonia is fixed (600 ML) in all the ratios, other ratios has less CH$_4$ and this problem is less serious. This is also not observed in the ice mixtures after irradiation. Using the same reference spectrum, that is the infra-red spectrum recorded before deposition, we plotted the infra-red spectrum of irradiated ice mixture in figure 3.1.2. It shows the absorbance of each ratio after irradiation. We labelled the peaks which we used to calculate the column densities onto the graph. Main products we have detected are C$_2$H$_6$, CN$^-$ and C$_3$H$_8$. The peak positions with the references are listed at \ref{tab:WavenumberMDHL}.\\

\begin{table}[htbp]
\begin{tabular}{ccccccc}
\hline
\hline
\multicolumn{2}{c}{Literture assignments} & \multicolumn{4}{c}{CH$_4$+NH$_3$ ratio (MDHL)} &  \\
\hline
Wavenumber (cm$^{-1}$) & Carrier  & 1:5 (cm$^{-1}$) & 1:10 (cm$^{-1}$) & 1:20 (cm$^{-1}$) & 3:2 (cm$^{-1}$) & Reference \\
3375 & $\nu_3$ (NH$_3$) & 3366 & 3366 & 3369 & 3367 & 1 \\
3290 & $2\nu_4$ (NH$_3$) & - & - & - & - & 1 \\
3210 & $\nu_1$ (NH$_3$) & 3207 & 3208 & 3210 & 3205 & 1 \\
\hline
\end{tabular}
\caption{The peak positions of identified substances after irradiation in different configurations of ice mixtures.}
\label{tab:WavenumberMDHL}
\end{table}

From infra-red absorption spectrum and their positions, we assigned the peak 2086 cm$^{-1}$ to CN$^-$  but not a combination of HCN and CN$^-$ because of we cannot observe the HCN bending mode located at 848 cm$^{-1}$. Although in the case CH$_4$ + NH$_3$ = 3:2, we may observe a peak located at 820 cm$^{-1}$, the peak is with a FWHM half of HCN and it disappeared at 50 K. Since 50 K is the desorbing temperature of C$_2$H$_6$ and the position is v12 mode of C$_2$H$_6$, we assign the 820 cm$^{-1}$ peak as C$_2$H$_6$. As the absence of HCN bending mode, we may assign our peak located at 2086 cm$^{-1}$ as CN$^-$. Other assignments such as C$_2$H$_6$ and C$_3$H$_8$ are observed with multiple peaks.\\
We integrated the area and divided by the absorption strength stated in table 3.2. Although we understand that there is an average error in absorption strengths of no more than 10 %  when the pure ice is diluted in N$_2$ and H$_2$O Richey & Gerakines 2012). In our case, absorption strengths changes after CH$_4$ and NH$_3$ are mixed. For example, according to d' Hendecourt & Allamandola (1986), the band of NH$_3$ located at 1070 cm$^{-1}$ would not change much (from $1.1 \times 10^{-17}$ to $1.2 \times 10^{-17}$) when excess water is added to pure NH$_3$ and therefore, we may use the same absorption strength throughout our discussion to give a brief concept on what is the column density of the species and how is the absorption area changes when concentrations of ice mixtures and photon energy are changed. For the case of CN$^-$, we know that CN$^-$ has a bond order =3 by its molecular orbitals which is different from CN stretching (bond order 2.5) which is very sensitive to the matrix environment. As an example, by Borget et al. (2012), the CN stretch in amino acetonitrile change by factor of 2 between the pure molecule itself and in a mixture of amino acetonitrile and H2O (1:3). Here, we adopt the absorption strengths stated in table 3.2 and neglect the error in absorption strengths.\\
