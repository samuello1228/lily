\begin{abstracten}
\index{ncuthesis 環境!abstracten}

{\bf \sf Keywords:} interstellar ice, Charon, VUV irradiation, EUV irradiation

\vspace{2em}

We never stop the exploration of the outer space. Since 1900s, astronomers have observed all over the sky to seek origin of life. Apart from observing stars, we may simulate the outer space environments and make some simple molecules in laboratories on the earth now. To investigate the formation of CN$^-$, we deposite CH$_4$ and NH$_3$ (mechanism proposed by Kim and Kaiser (2011)\cite{kim}) to simulate the surface of Charon. We provide VUV and EUV irradiations as energy sources and mainly use Fourier Transform Infrared Spectrometer (FTIR) and Quadrupole Mass Spectrometer (QMS) to study different concentrations of CH$_4$ to NH$_3$ ice mixtures.

\end{abstracten} 

