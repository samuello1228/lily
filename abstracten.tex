\begin{abstracten}
\index{ncuthesis 環境!abstracten}

{\bf \sf Keywords:} interstellar ice, Charon, VUV irradiation, EUV irradiation

\vspace{2em}

We never stopped the exporation of the outer space. Astronomers observed all over the sky to seek origin of life. Apart from observing stars, we may simulate the outer space environments and make some simple molecules in laboratories on earth. In 2002, a group of Japanese and US scientists (Miyakawa et al. (2002)\cite{miyakawa2002cold}) used NH$_4$CN, which was placed in -78 $^o$C for 23 years, and discovered adenine. CN$^-$ is believed as the origin of amino CN group. To investigate the formation of CN$^-$, we deposited CH$_4$ and NH$_3$ (mechanism proposed by Kim and Kaiser (2011)\cite{kim}) at 14 K and 1 $\times$ 10$^{-11}$ torr to demonstrate the surface of Charon. We provided VUV and EUV irradiations as energy sources and mainly used Fourier Transform Infrared Spectrometer (FTIR) and Quadrupole Mass Spectroscopy (QMS) to study different concentrations of CH$_4$ to NH$_3$ ice mixtures.


\end{abstracten} 

