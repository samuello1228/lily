\begin{abstractcn}
\index{ncuthesis 環境!abstractcn}

關鍵字:星際冰晶,冥衛一,真空紫外光,極紫外光
\vspace{2em}

自二十世紀以來,科學家藉著天文觀測技術的進步希望能從太空中尋找出與地球生命起源相關的脈絡。除了天文觀測外,如今在實驗室中模擬外太空的環境也已可行,透過實驗室中的模擬太空中極端的化學反應已成為目前瞭解地外化學環境的研究風潮。本論文主要的目的是希望瞭解如何透過光激發一些簡單的無機分子混合冰來形成CN化學鍵的鍵結生成,此為生成氨基酸前必要的化學鍵結。我們在低溫下製備了一系列不同濃度CH$_4$和NH$_3$的混合冰晶,利用真空紫外光及極紫外光照射這些不同比例混合的冰晶樣品並且透過測量其紅外光譜及質譜的變化而有系統的研究CN$^-$及CH$_3$NH$_2$的生成機制。我們亦將實驗結果與Kim和Kaiser(2011)\cite{kim}之前以5000 eV的電子照射結果做比對討論光激發與電子輻射激發機制的不同。我們的實驗結果除了可以用來模擬冥衛一(Charon)表面的化學反應外也提供了在冥衛一的經過冬季後偵測CN$^-$的可能性。人們對於外太空的好奇與探索正如我們的宇宙一樣永無止境。


\end{abstractcn} 
