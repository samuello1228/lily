\begin{abstractcn}
\index{ncuthesis 環境!abstractcn}

關鍵字:星際冰晶,冥衛一,真空紫外光,超真空紫外光
\vspace{2em}

\textbf{我們永遠不會停止對外太空的探索。自二十世紀以來,天文學家們一直在天空中觀察生命的起源。除了觀測恆星以外,我們現在可以模擬外太空環境,在地球上的實驗室中製造一些簡單的分子。為了研究CN$^-$的形成,我們將CH$_4$和NH$_3$(由Kim和Kaiser(2011)\cite{kim}提出的機制)進行沉積以模擬冥衛一(Charon)表面。我們提供VUV和EUV輻射作為能源,主要使用傅里葉變換紅外光譜(FTIR)和四極桿質譜(QMS)來研究不同濃度的CH$_4$和NH$_3$冰混合物。}

\end{abstractcn} 
