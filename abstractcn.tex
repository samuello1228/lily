\begin{abstractcn}
\index{ncuthesis 環境!abstractcn}

關鍵字:星際冰晶,冥衛一,真空紫外光,超真空紫外光
\vspace{2em}

人類從未停止對外太空的探索。為了尋找生命的起源,天文學家們觀測了一個又一個的星球。然而,除了觀測星球外,我們還能在實驗室模擬外太空的狀態,並把星際中的一些簡單份子製作出來。在實驗室模擬星際中的環境來達到探索生命起源的目的。在2002年,一群日本和美國科學家MIYAKAWA et al. (2002)\cite{miyakawa2002cold}在被稀釋的放置在-78度的環境中23年的NH$_4$CN中發現了胺基酸的一種(adenine), 而當中CN$^-$正是胺基酸CN的來源。為了探討CN$^-$ 的生成,本文使用CH$_4$和NH$_3$在真空環境(1 $\times$ 10$^{-11}$ torr)和非常低溫(15 K)的混合物, 來模擬太陽系中的冥衛一(Charon)的表面。 我們使用真空紫外光(VUV)和超真空紫外光(EUV)來模擬太陽系中的能量來源並使用傅里葉紅外光譜儀和四極質譜儀來探討CN$^-$的生成。

\end{abstractcn} 
