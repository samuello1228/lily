\begin{abstractcn}
\index{ncuthesis 環境!abstractcn}

關鍵字:星際冰晶,冥衛一,真空紫外光,超真空紫外光
\vspace{2em}

\textbf{我們永遠不會停止探索外太空的,自二十世紀以來,天文學家們不斷找尋生命的起源。除了觀測恆星以外,我們現在可以在實驗室中模擬外太空環境,製造一些簡單的分子。為了研究CN−的形成,我們長CH$_4$和NH$_3$的冰晶(由Kim和Kaiser(2011)\cite{kim}提出的機制)以模擬冥衛一(Charon)表面。我們提供真空紫外輻射(VUV)和極紫外輻射(EUV)輻射作為光源,主要使用傅里葉變換紅外光譜儀(FTIR)和四極質譜儀(QMS)來研究不同濃度的CH$_4$和NH$_3$冰混合物。}


\end{abstractcn} 
