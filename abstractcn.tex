\begin{abstractcn}
\index{ncuthesis 環境!abstractcn}

關鍵字:星際冰晶,冥衛一,真空紫外光,超真空紫外光
\vspace{2em}

人類從未停止對外太空的探索。為了尋找生命的起源,天文學家們觀測了一個又一個的星球。然而,除了觀測星球外,我們還能在實驗室模擬外太空的狀態,並把星際中的一些簡單分子製作出來。在實驗室模擬星際中的環境來達到探索生命起源的目的。為了探討CN$^-$ 的生成,本文使用CH$_4$和NH$_3$在真空環境(1 $\times$ 10$^{-10}$ torr)和非常低溫(15 K)的混合物, 來模擬太陽系中的冥衛一(Charon)的表面。 我們使用真空紫外輻射(VUV)和極紫外輻射(EUV)來模擬太陽照射在冥衛表面,並使用傅里葉紅外光譜儀和四極質譜儀來探討CN$^-$的生成。

\end{abstractcn} 
