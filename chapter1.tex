\chapter{Introduction}


According to Hindu cosmological mythology, ancient people believe that a giant turtle bears the world on its back. Even after we stepped onto the moon at 1969, there are still plenty that we cannot explain. In the novel Lord of the Rings, the author named the path between hobbits as Mordor, which is also the name of the dark area on Pluto’s moon, Charon. Recently, Mission New Horizons retrieved valuable data about Charon and Pluto. This thesis aims to explain the formation mechanisms of the red cap on the pole of Charon (fig. 1), especially during the long cold dark period, through observations in extreme ultra-violet (EUV) and vacuum ultra-violet (VUV) irradiation.\\
Composition of Charon\\
The main composition on the surface of Charon is H$_2$O. According to Infrared (IR) spectroscopy, it is a mixture of 90 \% H$_2$O and 10 \% tholin at millimetre depth. The second most dominant component is ammonia hydrate, which can be observed by earth-based telescopes (brown 2000, cook 2007). In far IR spectrum taken by LEISA camera on the New Horizons, concentrated ammonia is found on Organa crater (fig 2.) and throughout Charon (fig 3.) (Grundy 2016a). The third component which forms the dark red cap (tholin?) is cold-trapped methane from Pluto’s atmosphere ejecta (Hoey 2017). The presence of nitrogen and other ejecta from Pluto are neglected in this thesis because according to the model of Hoey et al. (2017) (fig.4), during New horizons’ approach, 98 \% of the arrived ejecta is CH$_4$. Charon’s atmospheric pressure is further constrained by New Horizons to be below 0.3 nano bars, which is $4 \times 10^{-13}$ torr for all 14 atoms and molecules including CO, H$_2$, CH$_4$, Ne, Ar, etc. (fig. 5). CH$_4$ remains undetectable when we convert the momentum of CH$_4$ with 7 hops on the surface of Charon until deposited onto cold enough part is $1 \times 10^{-11}$ Pa, which is $7.5 \times 10^{-14}$ torr (Grundy 2016b).\\
VUV irradiation\\
Ly-α appears to be the largest source in the dark side of Charon, with attributions from both solar occultation (70 \%) and resonance scattering by atomic hydrogen flow (30 \%) in the solar system at flux $3.5 \times 10^7$ photons cm$^{-2}$ s$^{-1}$ onto the winter pole of Charon (Grundy 2016b). The flux is 50 \% larger than expected before Mission New Horizons (Gladstone 2015). CH$_4$ deposits at temperature below 25 K at pressure $7.4 \times 10^{-14}$ torr. The time for depositing CH$_4$ is 2 times longer at the pole (130 earth years) than at 45˚ lattitude according to the thermal model of Grundy et al. (2016b) (fig 6). In order to understand the formation of tholin at different latitudes of Charon, we performed VUV irradiation on CH$_4$+NH$_3$ and CH$_4$+NH$_3$+H$_2$O experiments with different ratios (including 3:2, 1:5, 1:10 and 1:20 for CH$_4$+NH$_3$ and 5:3:4, 1:5:5 and 1:10:10 for CH$_4$+NH$_3$+H$_2$O ice mixtures) to simulate the conditions at different latitudes on Charon with base pressure $3 times 10^{-10}$ torr, simulating atmosphere on Charon at 15 K, which corresponds to temperature on Charon at winter times (Grundy 2016b) in interstellar processing system (IPS) (Chen 2014).\\
EUV irradiation\\
Apart from VUV irradiation, EUV irradiation also took part. VUV irradiation is believed to be the main process to convert CH$_4$ into heavier molecules which remained on the surface of Charon until the temperature of Charon become 60 K, at which methane evaporates from the ice. The ice is then further processed by EUV, solar wind, coronal mass ejections and interstellar pickup ions, etc to produce the tholin on Charon (Grundy 2016b). The EUV irradiation (>12.4 eV) is $8.7 \times 10^7 eV cm^{-2} s^{-1}$ at mean heliocentric distance 39 A.U. whereas VUV irradiation (Ly-α) is $1.9 \times 10^9 eV cm^{-2} s^{-1}$. In order to investigate the effectiveness of EUV to VUV irradiation, we kept temperature of CH$_4$+NH$_3$ (3:2 \& 1:5) and CH$_4$+NH$_3$+H$_2$O (5:3:4) ice mixtures at 15 K and use the monochromatic 30.4 nm (He II) light provided by High flux beamline at National Synchrotron Radiation Research Centre (NSRRC) in Taiwan to irradiate the ice mixtures.\\
H$_2$O involved?\\
We compared the conditions of CH$_4$+NH$_3$ and CH$_4$+NH$_3$+H$_2$O because tholin on Titan is believed to be formed by CH$_4$+N$_2$ and a similar colour was observed on Charon. Charon is different from Titan as H$_2$O dominates on Charon. What are the differences between tholin formed by CH$_4$+NH$_3$ and CH$_4$+N$_2$? What role does H$_2$O play on Charon in the formation of tholin? Is it just diluting the formation or new compounds are formed?\\
In this thesis, we will introduce the formation reaction mechanisms of CH$_4$+NH$_3$ ice mixtures in EUV and VUV irradiation (section 3), the formation reaction mechanisms of CH$_4$+NH$_3$+H$_2$O ice mixtures in EUV and VUV irradiation (section 4), and the residues of these mixtures and a brief comparison with tholin on Titan will be made (section 5). With these results, we will have a better understanding about Charon and some astrophysical implications will be presented (section 6).\\
