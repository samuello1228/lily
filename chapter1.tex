\chapter{\protect Introduction}
\label{introduction}

According to Hindu cosmological mythology, ancient people believe that a giant turtle bears the world on its back. Even after we stepped onto the moon at 1969, there are still plenty that we cannot explain. Recently, a group of scientists put a dilute NH$_4$CN in temperature of liquid nitrogen for 27 years and discovered an amino-acid : adenine \cite{miyakawa2002cold}. NH$_4^+$CN$^-$ plays an important role in life evolution. The formation of CN$^-$ is proposed by Kim and Kaiser (2001) \cite{kim},which is produced by ammonia (NH$_3$) and methane (CH$_4$). However, they have only demonstrated the effects of cosmic rays (energetic electrons) onto the ice mixtures, the photolysis experiments of CH$_4$+NH$_3$ ice mixtures are still not well understood. This thesis aims to investigate the chemistry of VUV and EUV irradiations on CH$_4$+NH$_3$ ice mixtures, which is possibly one of the main starting components to form CN$^-$ in astrophysical environments.\\

NH$_3$ is often not probed in astrophysical environments unless detecting aimbiguously. Nearly all the infra-red bands overlap with water. The "unbrella" mode (1070 cm$^{-1}$) is often obscured by the 10 micron silicate feature \cite{d1986time}. It is often detected as ammonia hydrates (water $-$ ammonia mixtures)\cite{cook2007near}. The New Horizons team has revealed a high concentration crater of NH$_3$\cite{grundy2016surface}(figure \ref{fig:Charon_IR} on one of the icy satellites, Charon. In the graph, the green pigments indicates the concentration of ammonia on the left topological graph of Charon. The infrared spectra detected by LEISA camera regarding four different segments are presented at the right panel. On Organa crater, we may observe a 2.2 $\mu$m absorption representing presence of ammonia at "b" spectrum. This also explains the different concentration of ammonia detected by different earth-based observation groups.\cite{cook2007near},\cite{brown2000evidence}\\

Charon, as a member in the Pluto system, is the second massive member with masses about half of Pluto. They orbits around the Sun with an semi-major axis of 39.1 A.U. and period of 248 earth years. Similar to the earth and moon system, they orbits synchronized to each other with tidal-lockings. Therefore, the same side always facing each other. CH$_4$ is also present on Charon, which is the main ejecta arrived from Pluto to Charon. Hoey et al. (2017)\cite{hoey2017rarefied} used gravitational potential field models to simulate the distributed ejectas of Pluto, 98 \% of them are CH$_4$, hits on Charon during New horizons' fly-by (figure \ref{fig:Charon_distribution}). With his model, we may calculate the amount of methane deposited onto Charon during the winter time, which is presented in chapter \ref{astron}.\\

The non-detection of methane by infrared spectra implies that its concentration is less than ammonia. However, the exact concentration ratios to ammonia was not fully modeled yet. Therefore, we decided to perform 3 ratios of CH$_4$ to NH$_3$ (in excess) to simulate the surface of Charon. They includes CH$_4$ to NH$_3$ equals 1:5, 1:10 and 1:20. The ratios in previous studies with electron irradiation experiments are CH$_4$ to NH$_3$ equals 3:1\cite{kim} and 3:2\cite{kundu2017electron}. As a complete study, we decided performing a ratio of CH$_4$ to NH$_3$ equals 3 to 2 to make possible comparisons.\\

\begin{figure}
\centering
\includegraphics[width=\textwidth]{figures/chapter1/IR.png}
\caption{The 2.2$\mu$m absorption taken by LEISA camera colored as green on the topology shown by LORRI camera (A) and the spectra at 4 positions (B) with b taken near organa crater.(quoted from \cite{grundy2016surface})}
\label{fig:Charon_IR}
\end{figure}

\begin{figure}
\centering
\includegraphics[width=\textwidth]{figures/chapter1/methane.png}
\caption{The simulation of N$_2$ and CH$_4$ model assuming all arrived molecules will stick onto the surface of Charon. Among the deposition rate, 98 \% of them are CH$_4$ because CH$_4$ is lighter and preferentially escapes. The molar fraction of CH$_4$ increase from hypothesized 0.44 \% to 42 \% in the exobase of Pluto.(quoted from \cite{hoey2017rarefied})}
\label{fig:Charon_distribution}
\end{figure}

Despite methane and ammonia, we still need energy to generate CN$^-$. In our solar system, there are many energetic sources, including solar wind, photons, cosmic rays from the outer solar system, etc. Among these, Ly-$\alpha$ appears to be the most intense source in the dark side of Charon. It is attributed from direct sunlight(70 \%) and resonance scattering by atomic hydrogen flow (the excitment of electrons in hydrogen atoms presents in solar system and release ly-$\alpha$ in all direction)(30 \%) in the solar system \cite{grundy2016formation}. Its flux is $3.5 \times 10^7$ photons cm$^{-2}$ s$^{-1}$ at the winter pole of Charon \cite{grundy2016formation} which is 50 \% larger than expected before Mission New Horizons \cite{gladstone2015lyalpha}. We perform VUV irradiation on CH$_4$+NH$_3$ experiments with different ratios (including 3:2, 1:5, 1:10 and 1:20) to simulate the effect of Ly-$\alpha$ on different concentrations of CH$_4$, which deposits at temperature below 25 K at pressure $7.4 \times 10^{-14}$ torr onto the surface of Charon. The mean cold-trap longitivity for depositing CH$_4$ is 2 times longer at the poles (130 earth years) than at 45$^{\circ}$ lattitude \cite{grundy2016formation} (figure \ref{fig:Charon_thermal}).\\

\begin{figure}
\centering
\includegraphics[width=0.5\textwidth]{figures/chapter1/thermal.png}
\caption{The temperature of Charon with thermal inertia 10 J m$^{-2}$ K$^{-1}$ s$^{-1/2}$ in 1750 to 2050 Earth years (a) and longest time the Latitude is under 25 K with the model averaged for 3 Myr with 2.5 (solid) 10 (dotted) and 40 (dashed) J m$^{-2}$ K$^{-1}$ s$^{-1/2}$ (b).(quoted from \cite{grundy2016formation})}
\label{fig:Charon_thermal}
\end{figure}

Apart from VUV irradiation, EUV irradiation also irradiates on Charon. The EUV irradiation (>12.4 eV) is $8.7 \times 10^7$ eV cm$^{-2}$ s$^{-1}$ at mean heliocentric distance 39 A.U. whereas VUV irradiation (Ly-$\alpha$) is $1.9 \times 10^9$ eV cm$^{-2}$ s$^{-1}$\cite{grundy2016formation}. In order to investigate the effectiveness of EUV to VUV irradiation, we keep temperature of CH$_4$+NH$_3$ (3:2 \& 1:5) ice mixtures at 15 K and use the monochromatic 30.4 nm (40.8 eV) (He II) light provided by High flux beamline at National Synchrotron Radiation Research Centre (NSRRC) in Taiwan to irradiate the ice mixtures.\\

In this text, we will introduce the experimental methodology in chapter \ref{methods}, the formation mechanisms of main products of  EUV and VUV irradiated CH$_4$+NH$_3$ ice mixtures\ref{results}. With these results, we will know more details of Charon, especially the influences of photon sources. Different energy sources including electron irradiation experiments , EUV and VUV irradiations, and their astrophysical implications will be presented in chapter \ref{astron}.\\
