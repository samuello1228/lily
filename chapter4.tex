\chapter{\protect Astrophysical Implications}
\label{astron}

In this text, we look at effects
The main source to irradiating the dark side of Charon is Ly$\alpha$ reflected by interplanetary medium \cite{grundy2016formation}. Other sources included energetic ions in solar wind, which mainly consist of H$^+$, He$^+$, He$^{++}$ and O$^{2+}$ etc. originated from solar corona or IPM. These ions also reflect solar irradiation to the dark side of Charon. Among sources focused on He II irradiation as it is 3 to�� 20 times more intense than He I during a solar flare. As the intensity varies with solar activities, it is difficult to estimate the dose onto Charon. Besides, electronic flux is also present in solar wind but it is one order of magnitude lower than proton flux. The flux for energetic electrons observed at the 1 A. U. position is available (http://www.swpc.noaa.gov/products/goes-electron-flux). Although electron flux is much less important than Ly-$\alpha$, and their flux varies, we also compare the electron irradiation experiment done by Kim and Kaiser \cite{kim} on CH$_4$+NH$_3$ ice mixtures.


\section{Cyanide ion produced by photon source and electron source} %comparison with Kim use beer's law to get the total irradiated ice 

We study the ice mixtures in which CH$_4$ is dominated and compare the efficiencies in CN$^-$ formation by electrons and VUV irradiations.  Comparing our CN$^-$ obtained after infinitely long exposure, 13 – 16 ML of CN$^-$ was obtained by electron irradiation depending on which equation they choose to fit. In our MDHL experiments, we have 14.8 ML of CN$^-$. However, Kim and Kaiser (2011) adopted the CN$^-$ absorption coefficient measured by Georgieva and Velcheva (2006) to be 3.7 $\times$ 10$^{-18}$ cm molecule$^{-1}$, which is 4.86 times smaller. We do not adopt this absorption coefficient because it violates the carbon balance that number of CN$^-$ produced will be larger than CH$_4$ consumption. If we adopted the same absorption coefficient, the production yield of CN$^-$ should be multiplied by 4.86. Therefore, our yield is 72 ML of CN$^-$. Regarding percentage of NH$_3$ (limiting reactant), Kim and Kaiser has 5 - 6 \% yield where we have 12 \% yield if we adopted the same absorption coefficients. 

Note that Kim and Kaiser \cite{kim} have ice thickness (CH$_4$ = 610 ML, NH$_3$= 260 ML) while we have thicker ices (CH$_4$ = 900 ML, NH$_3$= 600 ML). Therefore, we calculate the percentage of photons absorbed by CH$_4$ and NH$_3$ ice mixtures under VUV irradiation. Substituting cross-sections measured by Cruz-Diaz et al. \cite{cruz2014vacuum} and the spectrum of our MDHL and substitute them into Beer$'$s law, CH$_4$+NH$_3$ = 3:2 ice mixtures can absorb more than 99 \% of light when thickness of CH$_4$ and NH$_3$ equals % ML. Therefore, we may assume all the irradiated light is absorbed by the ice. For CH$_4$+NH$_3$ =3:2 ice mixture, around 9 $\times$ 10$^{17}$ photons were irradiated in 270 minutes.

In electron irradiation experiments of Kim and Kaiser (2011)\cite{kim}, the energy transferred to CH$_4$ + NH$_3$ ice mixtures is by linear electron transfer (LET) that 1.3 eV molecule$^{-1}$ was absorbed by the ice in 90 minutes. They get flattened at 20 minutes’ irradiation, with fluence of 2.0 $\times$ 10$^{14}$ electrons cm$^{-2}$. While we got flattened at a dose of 3 $\times$ 10$^{17}$ photons cm$^{-2}$. Considering the energy of their electron (1.5 keV) and energy of our photons, they got flattened at 3 $\times$ 10$^{17}$ eV cm$^{-2}$ while we get flattened at 27.81 $\times$ 10$^{17}$ eV cm$^{-2}$. Comparing these energy doses, less electrons are needed to flatten the formation of CN$^-$.

To conclude, electron irradiation has a smaller absorption cross-sections, the percentage of yield is also smaller than VUV irradiated ice mixtures with similar ice thicknesses.




